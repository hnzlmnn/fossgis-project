% !TeX root = ../document.tex

\noindent\textbf{Abstract.} The "Intra-Urban Heat Island Effect" describes the phenomenon that certain urban areas experience higher heat accumulation - und thus temperatures - over the course of a day than other areas, such as natural landscapes or open water due to the absorption and re-emittance of infrastructure such as buildings, roads and artificial or sealed surfaces.

Increased heat islands within the city have a negative effect, not only on ecosystems, but on people’s health and well-being especially for the vulnerable populations e.g. infants and elderly people. Hence, the heat islands effect should be considered during city planning activities.

The goal for this project is to analyze temperature data in Berlin to identify possible locations of heat islands and their development during a day. Raw data was obtained from the citizen science project openSenseMap\footnote{\url{https://opensensemap.org/}}, aggregated into ten-minute average measurements and used to interpolate a series of stations within the boundaries of Berlin. The final product - an animation of the temperature over the course of a day - will be beneficial for planning authorities and in developing measures to counteract urban heat islands.
